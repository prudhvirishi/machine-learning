\documentclass[]{article}
\usepackage{lmodern}
\usepackage{amssymb,amsmath}
\usepackage{ifxetex,ifluatex}
\usepackage{fixltx2e} % provides \textsubscript
\ifnum 0\ifxetex 1\fi\ifluatex 1\fi=0 % if pdftex
  \usepackage[T1]{fontenc}
  \usepackage[utf8]{inputenc}
\else % if luatex or xelatex
  \ifxetex
    \usepackage{mathspec}
  \else
    \usepackage{fontspec}
  \fi
  \defaultfontfeatures{Ligatures=TeX,Scale=MatchLowercase}
\fi
% use upquote if available, for straight quotes in verbatim environments
\IfFileExists{upquote.sty}{\usepackage{upquote}}{}
% use microtype if available
\IfFileExists{microtype.sty}{%
\usepackage{microtype}
\UseMicrotypeSet[protrusion]{basicmath} % disable protrusion for tt fonts
}{}
\usepackage[margin=1in]{geometry}
\usepackage{hyperref}
\hypersetup{unicode=true,
            pdftitle={Lab2Block2},
            pdfauthor={Prudhvi Peddmallu},
            pdfborder={0 0 0},
            breaklinks=true}
\urlstyle{same}  % don't use monospace font for urls
\usepackage{color}
\usepackage{fancyvrb}
\newcommand{\VerbBar}{|}
\newcommand{\VERB}{\Verb[commandchars=\\\{\}]}
\DefineVerbatimEnvironment{Highlighting}{Verbatim}{commandchars=\\\{\}}
% Add ',fontsize=\small' for more characters per line
\usepackage{framed}
\definecolor{shadecolor}{RGB}{248,248,248}
\newenvironment{Shaded}{\begin{snugshade}}{\end{snugshade}}
\newcommand{\KeywordTok}[1]{\textcolor[rgb]{0.13,0.29,0.53}{\textbf{#1}}}
\newcommand{\DataTypeTok}[1]{\textcolor[rgb]{0.13,0.29,0.53}{#1}}
\newcommand{\DecValTok}[1]{\textcolor[rgb]{0.00,0.00,0.81}{#1}}
\newcommand{\BaseNTok}[1]{\textcolor[rgb]{0.00,0.00,0.81}{#1}}
\newcommand{\FloatTok}[1]{\textcolor[rgb]{0.00,0.00,0.81}{#1}}
\newcommand{\ConstantTok}[1]{\textcolor[rgb]{0.00,0.00,0.00}{#1}}
\newcommand{\CharTok}[1]{\textcolor[rgb]{0.31,0.60,0.02}{#1}}
\newcommand{\SpecialCharTok}[1]{\textcolor[rgb]{0.00,0.00,0.00}{#1}}
\newcommand{\StringTok}[1]{\textcolor[rgb]{0.31,0.60,0.02}{#1}}
\newcommand{\VerbatimStringTok}[1]{\textcolor[rgb]{0.31,0.60,0.02}{#1}}
\newcommand{\SpecialStringTok}[1]{\textcolor[rgb]{0.31,0.60,0.02}{#1}}
\newcommand{\ImportTok}[1]{#1}
\newcommand{\CommentTok}[1]{\textcolor[rgb]{0.56,0.35,0.01}{\textit{#1}}}
\newcommand{\DocumentationTok}[1]{\textcolor[rgb]{0.56,0.35,0.01}{\textbf{\textit{#1}}}}
\newcommand{\AnnotationTok}[1]{\textcolor[rgb]{0.56,0.35,0.01}{\textbf{\textit{#1}}}}
\newcommand{\CommentVarTok}[1]{\textcolor[rgb]{0.56,0.35,0.01}{\textbf{\textit{#1}}}}
\newcommand{\OtherTok}[1]{\textcolor[rgb]{0.56,0.35,0.01}{#1}}
\newcommand{\FunctionTok}[1]{\textcolor[rgb]{0.00,0.00,0.00}{#1}}
\newcommand{\VariableTok}[1]{\textcolor[rgb]{0.00,0.00,0.00}{#1}}
\newcommand{\ControlFlowTok}[1]{\textcolor[rgb]{0.13,0.29,0.53}{\textbf{#1}}}
\newcommand{\OperatorTok}[1]{\textcolor[rgb]{0.81,0.36,0.00}{\textbf{#1}}}
\newcommand{\BuiltInTok}[1]{#1}
\newcommand{\ExtensionTok}[1]{#1}
\newcommand{\PreprocessorTok}[1]{\textcolor[rgb]{0.56,0.35,0.01}{\textit{#1}}}
\newcommand{\AttributeTok}[1]{\textcolor[rgb]{0.77,0.63,0.00}{#1}}
\newcommand{\RegionMarkerTok}[1]{#1}
\newcommand{\InformationTok}[1]{\textcolor[rgb]{0.56,0.35,0.01}{\textbf{\textit{#1}}}}
\newcommand{\WarningTok}[1]{\textcolor[rgb]{0.56,0.35,0.01}{\textbf{\textit{#1}}}}
\newcommand{\AlertTok}[1]{\textcolor[rgb]{0.94,0.16,0.16}{#1}}
\newcommand{\ErrorTok}[1]{\textcolor[rgb]{0.64,0.00,0.00}{\textbf{#1}}}
\newcommand{\NormalTok}[1]{#1}
\usepackage{graphicx,grffile}
\makeatletter
\def\maxwidth{\ifdim\Gin@nat@width>\linewidth\linewidth\else\Gin@nat@width\fi}
\def\maxheight{\ifdim\Gin@nat@height>\textheight\textheight\else\Gin@nat@height\fi}
\makeatother
% Scale images if necessary, so that they will not overflow the page
% margins by default, and it is still possible to overwrite the defaults
% using explicit options in \includegraphics[width, height, ...]{}
\setkeys{Gin}{width=\maxwidth,height=\maxheight,keepaspectratio}
\IfFileExists{parskip.sty}{%
\usepackage{parskip}
}{% else
\setlength{\parindent}{0pt}
\setlength{\parskip}{6pt plus 2pt minus 1pt}
}
\setlength{\emergencystretch}{3em}  % prevent overfull lines
\providecommand{\tightlist}{%
  \setlength{\itemsep}{0pt}\setlength{\parskip}{0pt}}
\setcounter{secnumdepth}{0}
% Redefines (sub)paragraphs to behave more like sections
\ifx\paragraph\undefined\else
\let\oldparagraph\paragraph
\renewcommand{\paragraph}[1]{\oldparagraph{#1}\mbox{}}
\fi
\ifx\subparagraph\undefined\else
\let\oldsubparagraph\subparagraph
\renewcommand{\subparagraph}[1]{\oldsubparagraph{#1}\mbox{}}
\fi

%%% Use protect on footnotes to avoid problems with footnotes in titles
\let\rmarkdownfootnote\footnote%
\def\footnote{\protect\rmarkdownfootnote}

%%% Change title format to be more compact
\usepackage{titling}

% Create subtitle command for use in maketitle
\newcommand{\subtitle}[1]{
  \posttitle{
    \begin{center}\large#1\end{center}
    }
}

\setlength{\droptitle}{-2em}

  \title{Lab2Block2}
    \pretitle{\vspace{\droptitle}\centering\huge}
  \posttitle{\par}
    \author{Prudhvi Peddmallu}
    \preauthor{\centering\large\emph}
  \postauthor{\par}
      \predate{\centering\large\emph}
  \postdate{\par}
    \date{22 April 2019}


\begin{document}
\maketitle

\section{Assignment 1. Using GAM and GLM to examine the mortality
rates}\label{assignment-1.-using-gam-and-glm-to-examine-the-mortality-rates}

The Excel document influenza.xlsx contains weekly data on the mortality
and the number of laboratory-confirmed cases of influenza in Sweden. In
addition, there is information about population-weighted temperature
anomalies (temperature deficits).

\begin{enumerate}
\def\labelenumi{\arabic{enumi}.}
\tightlist
\item
  Use time series plots to visually inspect how the mortality and
  influenza number vary with time (use Time as X axis). By using this
  plot, comment how the amounts of influenza cases are related to
  mortality rates.
\item
  Use gam() function from mgcv package to fit a GAM model in which
  Mortality is normally distributed and modelled as a linear function of
  Year and spline function of Week, and make sure that the model
  parameters are selected by the generalized cross-validation. Report
  the underlying probabilistic model.
\item
  Plot predicted and observed mortality against time for the fitted
  model and comment on the quality of the fit. Investigate the output of
  the GAM model and report which terms appear to be significant in the
  model. Is there a trend in mortality change from one year to another?
  Plot the spline component and interpret the plot.
\item
  Examine how the penalty factor of the spline function in the GAM model
  from step 2 influences the estimated deviance of the model. Make plots
  of the predicted and observed mortality against time for cases of very
  high and very low penalty factors. What is the relation of the penalty
  factor to the degrees of freedom? Do your results confirm this
  relationship?
\item
  Use the model obtained in step 2 and plot the residuals and the
  influenza values against time (in one plot). Is the temporal pattern
  in the residuals correlated to the outbreaks of influenza? 6 Fit a GAM
  model in R in which mortality is be modelled as an additive function
  of the spline functions of year, week, and the number of confirmed
  cases of influenza. Use the output of this GAM function to conclude
  whether or not the mortality is influenced by the outbreaks of
  influenza. Provide the plot of the original and fitted Mortality
  against Time and comment whether the model seems to be better than the
  previous GAM models.
\end{enumerate}

\begin{Shaded}
\begin{Highlighting}[]
\KeywordTok{library}\NormalTok{(pamr)}
\KeywordTok{library}\NormalTok{(glmnet)}
\KeywordTok{library}\NormalTok{(dplyr)}
\KeywordTok{library}\NormalTok{(kernlab)}
\KeywordTok{library}\NormalTok{(ggplot2)}
\KeywordTok{library}\NormalTok{(akima)}
\KeywordTok{library}\NormalTok{(mgcv)}
\KeywordTok{library}\NormalTok{(readxl)}
\KeywordTok{library}\NormalTok{(grid)}
\KeywordTok{library}\NormalTok{(plotly)}
\KeywordTok{library}\NormalTok{(dplyr)}
\end{Highlighting}
\end{Shaded}

\begin{Shaded}
\begin{Highlighting}[]
\CommentTok{#data-file}
\KeywordTok{library}\NormalTok{(xlsx)}
\NormalTok{flu_data =}\StringTok{ }\KeywordTok{read.xlsx}\NormalTok{(}\StringTok{"influenza.xlsx"}\NormalTok{, }\DataTypeTok{sheetName =} \StringTok{"Raw data"}\NormalTok{)}
\KeywordTok{library}\NormalTok{(readxl)}
\NormalTok{influenza<-}\KeywordTok{read_xlsx}\NormalTok{(}\StringTok{"influenza.xlsx"}\NormalTok{)}
\end{Highlighting}
\end{Shaded}

\subsection{Question1.1-Time series plots to visually inspect how the
mortality and
influenza}\label{question1.1-time-series-plots-to-visually-inspect-how-the-mortality-and-influenza}

\begin{Shaded}
\begin{Highlighting}[]
\KeywordTok{library}\NormalTok{(ggplot2)}
\CommentTok{#plots}
\NormalTok{p<-}\KeywordTok{ggplot}\NormalTok{(}\DataTypeTok{data =}\NormalTok{ influenz, }\KeywordTok{aes}\NormalTok{(}\DataTypeTok{x =}\NormalTok{ Time, }\DataTypeTok{y =}\NormalTok{ Influenza)) }\OperatorTok{+}\StringTok{ }\KeywordTok{geom_line}\NormalTok{(}\KeywordTok{aes}\NormalTok{(}\DataTypeTok{color =} \StringTok{"#00AFBB"}\NormalTok{))}
\NormalTok{p}
\NormalTok{q<-}\KeywordTok{ggplot}\NormalTok{(}\DataTypeTok{data =}\NormalTok{ influenz, }\KeywordTok{aes}\NormalTok{(}\DataTypeTok{x =}\NormalTok{ Time, }\DataTypeTok{y =}\NormalTok{ Mortality)) }\OperatorTok{+}\StringTok{ }\KeywordTok{geom_line}\NormalTok{(}\KeywordTok{aes}\NormalTok{(}\DataTypeTok{color =} \StringTok{"#00AFBB"}\NormalTok{))}
\NormalTok{q}
\end{Highlighting}
\end{Shaded}

There is increase in influenza cases there is increase in mortality till
2000. After that influenza seems to be having lesser impact on
mortality.

\subsection{Question1.2-Fit a GAM
model}\label{question1.2-fit-a-gam-model}

\begin{Shaded}
\begin{Highlighting}[]
\NormalTok{gam_model=}\KeywordTok{gam}\NormalTok{(Mortality}\OperatorTok{~}\NormalTok{Year}\OperatorTok{+}\KeywordTok{s}\NormalTok{(Week,}\DataTypeTok{k=}\KeywordTok{length}\NormalTok{(}\KeywordTok{unique}\NormalTok{(influenz}\OperatorTok{$}\NormalTok{Week))),}\DataTypeTok{data=}\NormalTok{influenz,}\DataTypeTok{method=}\StringTok{"GCV.Cp"}\NormalTok{)}
\NormalTok{gam_model}
\NormalTok{(or)}
\NormalTok{res =}\StringTok{ }\KeywordTok{gam}\NormalTok{(Mortality }\OperatorTok{~}\StringTok{ }\NormalTok{Year }\OperatorTok{+}\KeywordTok{s}\NormalTok{(Week, }\DataTypeTok{k=}\KeywordTok{length}\NormalTok{(}\KeywordTok{unique}\NormalTok{(influenza}\OperatorTok{$}\NormalTok{Week))),}\DataTypeTok{method=}\StringTok{"GCV.Cp"}\NormalTok{, }\DataTypeTok{data=}\NormalTok{influenza)}
\KeywordTok{gam.check}\NormalTok{(res)}
\NormalTok{(or)}
\NormalTok{gam_model <-}\StringTok{ }\NormalTok{mgcv}\OperatorTok{::}\KeywordTok{gam}\NormalTok{(}\DataTypeTok{data =}\NormalTok{ flu_data, Mortality}\OperatorTok{~}\NormalTok{Year}\OperatorTok{+}\KeywordTok{s}\NormalTok{(Week), }\DataTypeTok{method =} \StringTok{"GCV.Cp"}\NormalTok{)}
\KeywordTok{summary}\NormalTok{(gam_model)}
\end{Highlighting}
\end{Shaded}

Analysis: Using the default parameter settings within the
\emph{gam}-function implies that \emph{Mortality} is normally
distributed (\emph{family=gaussian()}). Also, since \emph{method =
``GCV.Cp''}, this leads to the usage of GCV (\emph{Generalized Cross
Validation score}) related to the smoothing parameter estimation. The
underlying probabilistic model can be written as:
\[ Mortality = N(\mu, \sigma^2) \]
\[ \hat{Mortality} = Intercept + \beta_1Year + s(Week) + \epsilon     \]
where \[ \epsilon = N(0, \sigma^2) .\]

\subsection{Question1.3:-Plot predicted and observed mortality against
time for the fitted model and comment on the quality of the
fit.}\label{question1.3-plot-predicted-and-observed-mortality-against-time-for-the-fitted-model-and-comment-on-the-quality-of-the-fit.}

\begin{Shaded}
\begin{Highlighting}[]
\NormalTok{predicted <-}\StringTok{ }\KeywordTok{predict}\NormalTok{(res, }\DataTypeTok{newdata =}\NormalTok{ influenza, }\DataTypeTok{type=}\StringTok{'response'}\NormalTok{)}
\KeywordTok{ggplot}\NormalTok{(}\DataTypeTok{data =}\NormalTok{influenza) }\OperatorTok{+}
\KeywordTok{geom_point}\NormalTok{(}\KeywordTok{aes}\NormalTok{(Time,Mortality), }\DataTypeTok{colour =} \StringTok{"Blue"}\NormalTok{) }\OperatorTok{+}
\KeywordTok{geom_line}\NormalTok{(}\KeywordTok{aes}\NormalTok{(Time,predicted), }\DataTypeTok{colour =} \StringTok{"Red"}\NormalTok{)}
\KeywordTok{summary}\NormalTok{(res)}
\KeywordTok{plot}\NormalTok{(res)}\CommentTok{#Plot the spline component}
\end{Highlighting}
\end{Shaded}

Analysis:Significant terms in model - According to the p values in the
summary, Spline function of weekis the most sognificant term.Trend in
mortality change from one year to another - It seems to be following a
cyclic trendand seems to have decreased with the years.

\subsection{Question1.4:-the penalty factor of the spline function in
the GAM
model}\label{question1.4-the-penalty-factor-of-the-spline-function-in-the-gam-model}

\begin{Shaded}
\begin{Highlighting}[]
\NormalTok{res2 =}\StringTok{ }\KeywordTok{gam}\NormalTok{(Mortality }\OperatorTok{~}\StringTok{ }\NormalTok{Year }\OperatorTok{+}
\KeywordTok{s}\NormalTok{(Week, }\DataTypeTok{k=}\KeywordTok{length}\NormalTok{(}\KeywordTok{unique}\NormalTok{(influenza}\OperatorTok{$}\NormalTok{Week)), }\DataTypeTok{sp=}\DecValTok{0}\NormalTok{),}
\DataTypeTok{method=}\StringTok{"GCV.Cp"}\NormalTok{, }\DataTypeTok{data=}\NormalTok{influenza)}
\NormalTok{predicted2 <-}\StringTok{ }\KeywordTok{predict}\NormalTok{(res2, }\DataTypeTok{newdata =}\NormalTok{ influenza, }\DataTypeTok{type=}\StringTok{'response'}\NormalTok{)}
\NormalTok{p1 <-}\StringTok{ }\KeywordTok{ggplot}\NormalTok{(}\DataTypeTok{data =}\NormalTok{influenza) }\OperatorTok{+}
\KeywordTok{geom_point}\NormalTok{(}\KeywordTok{aes}\NormalTok{(Time,Mortality),}\DataTypeTok{colour =} \StringTok{"Blue"}\NormalTok{) }\OperatorTok{+}
\KeywordTok{geom_line}\NormalTok{(}\KeywordTok{aes}\NormalTok{(Time,predicted2),}\DataTypeTok{colour =} \StringTok{"Red"}\NormalTok{) }\OperatorTok{+}
\KeywordTok{ggtitle}\NormalTok{(}\StringTok{"sp = 0"}\NormalTok{)}
\NormalTok{res3 =}\StringTok{ }\KeywordTok{gam}\NormalTok{(Mortality }\OperatorTok{~}\StringTok{ }\NormalTok{Year }\OperatorTok{+}
\KeywordTok{s}\NormalTok{(Week, }\DataTypeTok{k=}\KeywordTok{length}\NormalTok{(}\KeywordTok{unique}\NormalTok{(influenza}\OperatorTok{$}\NormalTok{Week)), }\DataTypeTok{sp=}\FloatTok{0.01}\NormalTok{),}
\DataTypeTok{method=}\StringTok{"GCV.Cp"}\NormalTok{, }\DataTypeTok{data=}\NormalTok{influenza)}
\NormalTok{predicted3 <-}\StringTok{ }\KeywordTok{predict}\NormalTok{(res3, }\DataTypeTok{newdata =}\NormalTok{ influenza, }\DataTypeTok{type=}\StringTok{'response'}\NormalTok{)}
\NormalTok{p2 <-}\StringTok{ }\KeywordTok{ggplot}\NormalTok{(}\DataTypeTok{data =}\NormalTok{influenza) }\OperatorTok{+}
\KeywordTok{geom_point}\NormalTok{(}\KeywordTok{aes}\NormalTok{(Time,Mortality),}\DataTypeTok{colour =} \StringTok{"Blue"}\NormalTok{) }\OperatorTok{+}
\KeywordTok{geom_line}\NormalTok{(}\KeywordTok{aes}\NormalTok{(Time,predicted3),}\DataTypeTok{colour =} \StringTok{"Red"}\NormalTok{) }\OperatorTok{+}
\KeywordTok{ggtitle}\NormalTok{(}\StringTok{"sp = 0.01"}\NormalTok{)}
\NormalTok{res4 =}\StringTok{ }\KeywordTok{gam}\NormalTok{(Mortality }\OperatorTok{~}\StringTok{ }\NormalTok{Year }\OperatorTok{+}
\KeywordTok{s}\NormalTok{(Week, }\DataTypeTok{k=}\KeywordTok{length}\NormalTok{(}\KeywordTok{unique}\NormalTok{(influenza}\OperatorTok{$}\NormalTok{Week)), }\DataTypeTok{sp=}\DecValTok{1}\NormalTok{),}
\DataTypeTok{method=}\StringTok{"GCV.Cp"}\NormalTok{, }\DataTypeTok{data=}\NormalTok{influenza)}
\NormalTok{predicted4 <-}\StringTok{ }\KeywordTok{predict}\NormalTok{(res4, }\DataTypeTok{newdata =}\NormalTok{ influenza, }\DataTypeTok{type=}\StringTok{'response'}\NormalTok{)}
\NormalTok{p3 <-}\StringTok{ }\KeywordTok{ggplot}\NormalTok{(}\DataTypeTok{data =}\NormalTok{influenza) }\OperatorTok{+}
\KeywordTok{geom_point}\NormalTok{(}\KeywordTok{aes}\NormalTok{(Time,Mortality),}\DataTypeTok{colour =} \StringTok{"Blue"}\NormalTok{) }\OperatorTok{+}
\KeywordTok{geom_line}\NormalTok{(}\KeywordTok{aes}\NormalTok{(Time,predicted4),}\DataTypeTok{colour =} \StringTok{"Red"}\NormalTok{) }\OperatorTok{+}
\KeywordTok{ggtitle}\NormalTok{(}\StringTok{"sp = 1"}\NormalTok{)}
\KeywordTok{grid.newpage}\NormalTok{()}
\KeywordTok{grid.draw}\NormalTok{(}\KeywordTok{cbind}\NormalTok{(}\KeywordTok{ggplotGrob}\NormalTok{(p1), }\KeywordTok{ggplotGrob}\NormalTok{(p2), }\KeywordTok{ggplotGrob}\NormalTok{(p3), }\DataTypeTok{size =} \StringTok{"last"}\NormalTok{))}
\CommentTok{#anova(res2)}
\CommentTok{#some more models with different sp values}
\NormalTok{res5 =}\StringTok{ }\KeywordTok{gam}\NormalTok{(Mortality }\OperatorTok{~}\StringTok{ }\NormalTok{Year }\OperatorTok{+}
\KeywordTok{s}\NormalTok{(Week, }\DataTypeTok{k=}\KeywordTok{length}\NormalTok{(}\KeywordTok{unique}\NormalTok{(influenza}\OperatorTok{$}\NormalTok{Week)), }\DataTypeTok{sp=}\FloatTok{1.5}\NormalTok{),}
\DataTypeTok{method=}\StringTok{"GCV.Cp"}\NormalTok{, }\DataTypeTok{data=}\NormalTok{influenza)}
\NormalTok{res6 =}\StringTok{ }\KeywordTok{gam}\NormalTok{(Mortality }\OperatorTok{~}\StringTok{ }\NormalTok{Year }\OperatorTok{+}
\KeywordTok{s}\NormalTok{(Week, }\DataTypeTok{k=}\KeywordTok{length}\NormalTok{(}\KeywordTok{unique}\NormalTok{(influenza}\OperatorTok{$}\NormalTok{Week)), }\DataTypeTok{sp=}\DecValTok{2}\NormalTok{),}\DataTypeTok{method=}\StringTok{"GCV.Cp"}\NormalTok{, }\DataTypeTok{data=}\NormalTok{influenza)}
\NormalTok{res7 =}\StringTok{ }\KeywordTok{gam}\NormalTok{(Mortality }\OperatorTok{~}\StringTok{ }\NormalTok{Year }\OperatorTok{+}
\KeywordTok{s}\NormalTok{(Week, }\DataTypeTok{k=}\KeywordTok{length}\NormalTok{(}\KeywordTok{unique}\NormalTok{(influenza}\OperatorTok{$}\NormalTok{Week)), }\DataTypeTok{sp=}\DecValTok{3}\NormalTok{),}
\DataTypeTok{method=}\StringTok{"GCV.Cp"}\NormalTok{, }\DataTypeTok{data=}\NormalTok{influenza)}
\NormalTok{res8 =}\StringTok{ }\KeywordTok{gam}\NormalTok{(Mortality }\OperatorTok{~}\StringTok{ }\NormalTok{Year }\OperatorTok{+}
\KeywordTok{s}\NormalTok{(Week, }\DataTypeTok{k=}\KeywordTok{length}\NormalTok{(}\KeywordTok{unique}\NormalTok{(influenza}\OperatorTok{$}\NormalTok{Week)), }\DataTypeTok{sp=}\DecValTok{6}\NormalTok{),}
\DataTypeTok{method=}\StringTok{"GCV.Cp"}\NormalTok{, }\DataTypeTok{data=}\NormalTok{influenza)}
\NormalTok{res9 =}\StringTok{ }\KeywordTok{gam}\NormalTok{(Mortality }\OperatorTok{~}\StringTok{ }\NormalTok{Year }\OperatorTok{+}
\KeywordTok{s}\NormalTok{(Week, }\DataTypeTok{k=}\KeywordTok{length}\NormalTok{(}\KeywordTok{unique}\NormalTok{(influenza}\OperatorTok{$}\NormalTok{Week)), }\DataTypeTok{sp=}\DecValTok{9}\NormalTok{),}
\DataTypeTok{method=}\StringTok{"GCV.Cp"}\NormalTok{, }\DataTypeTok{data=}\NormalTok{influenza)}
\NormalTok{res10 =}\StringTok{ }\KeywordTok{gam}\NormalTok{(Mortality }\OperatorTok{~}\StringTok{ }\NormalTok{Year }\OperatorTok{+}
\KeywordTok{s}\NormalTok{(Week, }\DataTypeTok{k=}\KeywordTok{length}\NormalTok{(}\KeywordTok{unique}\NormalTok{(influenza}\OperatorTok{$}\NormalTok{Week)), }\DataTypeTok{sp=}\DecValTok{100}\NormalTok{),}
\DataTypeTok{method=}\StringTok{"GCV.Cp"}\NormalTok{, }\DataTypeTok{data=}\NormalTok{influenza)}
\DecValTok{2}\OperatorTok{-}\NormalTok{method}
\NormalTok{model_deviance <-}\StringTok{ }\OtherTok{NULL}
\ControlFlowTok{for}\NormalTok{(sp }\ControlFlowTok{in} \KeywordTok{c}\NormalTok{(}\FloatTok{0.001}\NormalTok{, }\FloatTok{0.01}\NormalTok{, }\FloatTok{0.1}\NormalTok{, }\DecValTok{1}\NormalTok{, }\DecValTok{10}\NormalTok{))}
\NormalTok{\{}
\NormalTok{k=}\KeywordTok{length}\NormalTok{(}\KeywordTok{unique}\NormalTok{(flu_data}\OperatorTok{$}\NormalTok{Week))}

\NormalTok{gam_model <-}\StringTok{ }\NormalTok{mgcv}\OperatorTok{::}\KeywordTok{gam}\NormalTok{(}\DataTypeTok{data =}\NormalTok{ flu_data, Mortality}\OperatorTok{~}\NormalTok{Year}\OperatorTok{+}\KeywordTok{s}\NormalTok{(Week, }\DataTypeTok{k=}\NormalTok{k, }\DataTypeTok{sp=}\NormalTok{sp), }\DataTypeTok{method =} \StringTok{"GCV.Cp"}\NormalTok{)}
\NormalTok{temp <-}\StringTok{ }\KeywordTok{cbind}\NormalTok{(gam_model}\OperatorTok{$}\NormalTok{deviance, gam_model}\OperatorTok{$}\NormalTok{fitted.values, gam_model}\OperatorTok{$}\NormalTok{y, flu_data}\OperatorTok{$}\NormalTok{Time_fixed,}
\NormalTok{              sp, }\KeywordTok{sum}\NormalTok{(}\KeywordTok{influence}\NormalTok{(gam_model)))}
\NormalTok{model_deviance <-}\StringTok{ }\KeywordTok{rbind}\NormalTok{(temp, model_deviance)}
\NormalTok{\}}
\NormalTok{model_deviance <-}\StringTok{ }\KeywordTok{as.data.frame}\NormalTok{(model_deviance)}
\KeywordTok{colnames}\NormalTok{(model_deviance) <-}\StringTok{ }\KeywordTok{c}\NormalTok{(}\StringTok{"Deviance"}\NormalTok{, }\StringTok{"Predicted_Mortality"}\NormalTok{, }\StringTok{"Mortality"}\NormalTok{, }\StringTok{"Time"}\NormalTok{,}
                              \StringTok{"penalty_factor"}\NormalTok{, }\StringTok{"degree_of_freedom"}\NormalTok{)}
\NormalTok{model_deviance}\OperatorTok{$}\NormalTok{Time <-}\StringTok{ }\KeywordTok{as.Date}\NormalTok{(model_deviance}\OperatorTok{$}\NormalTok{Time, }\DataTypeTok{origin =} \StringTok{'1970-01-01'}\NormalTok{)}
\CommentTok{# plot of deviance}
\NormalTok{p6 <-}\StringTok{ }\KeywordTok{ggplot}\NormalTok{(}\DataTypeTok{data=}\NormalTok{model_deviance, }\KeywordTok{aes}\NormalTok{(}\DataTypeTok{x =}\NormalTok{ penalty_factor, }\DataTypeTok{y =}\NormalTok{ Deviance)) }\OperatorTok{+}
\KeywordTok{geom_point}\NormalTok{() }\OperatorTok{+}
\StringTok{  }\KeywordTok{geom_line}\NormalTok{() }\OperatorTok{+}
\StringTok{      }\KeywordTok{theme_light}\NormalTok{() }\OperatorTok{+}
\KeywordTok{ggtitle}\NormalTok{(}\StringTok{"Plot of Deviance of Model vs. Penalty Factor"}\NormalTok{)}
\NormalTok{p6}
\CommentTok{# plot of degree of freedom}
\NormalTok{p7 <-}\StringTok{ }\KeywordTok{ggplot}\NormalTok{(}\DataTypeTok{data=}\NormalTok{model_deviance, }\KeywordTok{aes}\NormalTok{(}\DataTypeTok{x =}\NormalTok{ penalty_factor, }\DataTypeTok{y =}\NormalTok{ degree_of_freedom)) }\OperatorTok{+}
\KeywordTok{geom_point}\NormalTok{() }\OperatorTok{+}
\StringTok{  }\KeywordTok{geom_line}\NormalTok{() }\OperatorTok{+}
\StringTok{      }\KeywordTok{theme_light}\NormalTok{() }\OperatorTok{+}
\KeywordTok{ggtitle}\NormalTok{(}\StringTok{"Plot of degree_of_freedom of Model vs. Penalty Factor"}\NormalTok{)}
\NormalTok{p7}
\NormalTok{model_deviance_wide <-}\StringTok{ }\KeywordTok{melt}\NormalTok{(model_deviance[,}\KeywordTok{c}\NormalTok{(}\StringTok{"Time"}\NormalTok{, }\StringTok{"penalty_factor"}\NormalTok{,}
                                              \StringTok{"Mortality"}\NormalTok{, }\StringTok{"Predicted_Mortality"}\NormalTok{)],}
                            \DataTypeTok{id.vars =} \KeywordTok{c}\NormalTok{(}\StringTok{"Time"}\NormalTok{, }\StringTok{"penalty_factor"}\NormalTok{))}
\CommentTok{# plot of predicted vs. observed mortality}
\NormalTok{p8 <-}\StringTok{ }\KeywordTok{ggplot}\NormalTok{(}\DataTypeTok{data=}\NormalTok{model_deviance_wide[model_deviance_wide}\OperatorTok{$}\NormalTok{penalty_factor }\OperatorTok{==}\StringTok{ }\FloatTok{0.001}\NormalTok{,],}
             \KeywordTok{aes}\NormalTok{(}\DataTypeTok{x=}\NormalTok{ Time, }\DataTypeTok{y =}\NormalTok{ value)) }\OperatorTok{+}
\StringTok{  }\KeywordTok{geom_point}\NormalTok{(}\KeywordTok{aes}\NormalTok{(}\DataTypeTok{color =}\NormalTok{ variable), }\DataTypeTok{size=}\FloatTok{0.7}\NormalTok{) }\OperatorTok{+}
\StringTok{  }\KeywordTok{geom_line}\NormalTok{(}\KeywordTok{aes}\NormalTok{(}\DataTypeTok{color =}\NormalTok{ variable), }\DataTypeTok{size=}\FloatTok{0.7}\NormalTok{) }\OperatorTok{+}
\StringTok{  }\KeywordTok{scale_color_manual}\NormalTok{(}\DataTypeTok{values=}\KeywordTok{c}\NormalTok{(}\StringTok{"#E69F00"}\NormalTok{, }\StringTok{"#009E73"}\NormalTok{)) }\OperatorTok{+}
\StringTok{  }\KeywordTok{theme_light}\NormalTok{() }\OperatorTok{+}
\StringTok{  }\KeywordTok{ggtitle}\NormalTok{(}\StringTok{"Plot of Mortality vs. Time(Penalty 0.001)"}\NormalTok{)}
\NormalTok{p9 <-}\StringTok{ }\KeywordTok{ggplot}\NormalTok{(}\DataTypeTok{data=}\NormalTok{model_deviance_wide[model_deviance_wide}\OperatorTok{$}\NormalTok{penalty_factor }\OperatorTok{==}\StringTok{ }\DecValTok{10}\NormalTok{,],}
             \KeywordTok{aes}\NormalTok{(}\DataTypeTok{x=}\NormalTok{ Time, }\DataTypeTok{y =}\NormalTok{ value)) }\OperatorTok{+}
\StringTok{  }\KeywordTok{geom_point}\NormalTok{(}\KeywordTok{aes}\NormalTok{(}\DataTypeTok{color =}\NormalTok{ variable), }\DataTypeTok{size=}\FloatTok{0.7}\NormalTok{) }\OperatorTok{+}
\StringTok{    }\KeywordTok{geom_line}\NormalTok{(}\KeywordTok{aes}\NormalTok{(}\DataTypeTok{color =}\NormalTok{ variable), }\DataTypeTok{size=}\FloatTok{0.7}\NormalTok{) }\OperatorTok{+}
\StringTok{  }\KeywordTok{scale_color_manual}\NormalTok{(}\DataTypeTok{values=}\KeywordTok{c}\NormalTok{(}\StringTok{"#E69F00"}\NormalTok{, }\StringTok{"#009E73"}\NormalTok{)) }\OperatorTok{+}
\StringTok{    }\KeywordTok{theme_light}\NormalTok{() }\OperatorTok{+}
\StringTok{  }\KeywordTok{ggtitle}\NormalTok{(}\StringTok{"Plot of Mortality vs. Time(Penalty 10)"}\NormalTok{)}
\NormalTok{p8}
\NormalTok{p9}
\end{Highlighting}
\end{Shaded}

\subsection{Question1.5:-plot the residuals and the influenza values
against time (in one
plot).}\label{question1.5-plot-the-residuals-and-the-influenza-values-against-time-in-one-plot.}

\begin{Shaded}
\begin{Highlighting}[]
\NormalTok{edf is estimated degree of freedom }\ControlFlowTok{if}\NormalTok{ you see res2 u can get edf}
\NormalTok{edf vs sp}
\NormalTok{x <-}\StringTok{ }\KeywordTok{c}\NormalTok{(}\DecValTok{0}\NormalTok{, }\FloatTok{0.01}\NormalTok{, }\DecValTok{1}\NormalTok{, }\FloatTok{1.5}\NormalTok{, }\DecValTok{2}\NormalTok{, }\DecValTok{3}\NormalTok{, }\DecValTok{6}\NormalTok{, }\DecValTok{9}\NormalTok{, }\DecValTok{100}\NormalTok{)}
\NormalTok{y <-}\StringTok{ }\KeywordTok{c}\NormalTok{(}\KeywordTok{sum}\NormalTok{(res2}\OperatorTok{$}\NormalTok{edf), }\KeywordTok{sum}\NormalTok{(res3}\OperatorTok{$}\NormalTok{edf), }\KeywordTok{sum}\NormalTok{(res4}\OperatorTok{$}\NormalTok{edf), }\KeywordTok{sum}\NormalTok{(res5}\OperatorTok{$}\NormalTok{edf), }\KeywordTok{sum}\NormalTok{(res6}\OperatorTok{$}\NormalTok{edf),}
\KeywordTok{sum}\NormalTok{(res7}\OperatorTok{$}\NormalTok{edf), }\KeywordTok{sum}\NormalTok{(res8}\OperatorTok{$}\NormalTok{edf), }\KeywordTok{sum}\NormalTok{(res9}\OperatorTok{$}\NormalTok{edf), }\KeywordTok{sum}\NormalTok{(res10}\OperatorTok{$}\NormalTok{edf))}
\KeywordTok{data.frame}\NormalTok{(}\KeywordTok{cbind}\NormalTok{(}\DataTypeTok{sp=}\NormalTok{x,}\DataTypeTok{edf=}\NormalTok{y) )}
\KeywordTok{ggplot}\NormalTok{(}\DataTypeTok{data =}\NormalTok{ influenza, }\KeywordTok{aes}\NormalTok{(}\DataTypeTok{x =}\NormalTok{ Time)) }\OperatorTok{+}
\KeywordTok{geom_line}\NormalTok{(}\KeywordTok{aes}\NormalTok{(}\DataTypeTok{y =}\NormalTok{ Influenza,}\DataTypeTok{colour =} \StringTok{"Influenza"}\NormalTok{)) }\OperatorTok{+}
\KeywordTok{geom_line}\NormalTok{(}\KeywordTok{aes}\NormalTok{(}\DataTypeTok{y =}\NormalTok{ res}\OperatorTok{$}\NormalTok{residuals,}\DataTypeTok{colour =} \StringTok{"Residuals"}\NormalTok{)) }\OperatorTok{+}
\KeywordTok{scale_colour_manual}\NormalTok{(}\StringTok{""}\NormalTok{, }\DataTypeTok{breaks =} \KeywordTok{c}\NormalTok{(}\StringTok{"Influenza"}\NormalTok{, }\StringTok{"Residuals"}\NormalTok{),}
\DataTypeTok{values =} \KeywordTok{c}\NormalTok{(}\StringTok{"Red"}\NormalTok{, }\StringTok{"Blue"}\NormalTok{))}
\NormalTok{result}
\CommentTok{# sp edf}
\CommentTok{# 1 0.00 29.000000}
\CommentTok{# 2 0.01 6.689771}
\CommentTok{# 3 1.00 3.491752}
\CommentTok{# 4 1.50 3.365887}
\CommentTok{# 5 2.00 3.291922}
\CommentTok{# 6 3.00 3.208210}
\CommentTok{# 7 6.00 3.112130}
\CommentTok{# 8 9.00 3.076759}
\CommentTok{# 9 100.00 3.007268}
\CommentTok{#1.5}
\NormalTok{k=}\KeywordTok{length}\NormalTok{(}\KeywordTok{unique}\NormalTok{(flu_data}\OperatorTok{$}\NormalTok{Week))}
\NormalTok{gam_model <-}\StringTok{ }\NormalTok{mgcv}\OperatorTok{::}\KeywordTok{gam}\NormalTok{(}\DataTypeTok{data =}\NormalTok{ flu_data, Mortality}\OperatorTok{~}\NormalTok{Year}\OperatorTok{+}\KeywordTok{s}\NormalTok{(Week, }\DataTypeTok{k=}\NormalTok{k), }\DataTypeTok{method =} \StringTok{"GCV.Cp"}\NormalTok{)}
\NormalTok{temp <-}\StringTok{ }\NormalTok{flu_data}
\NormalTok{temp <-}\StringTok{ }\KeywordTok{cbind}\NormalTok{(temp, }\DataTypeTok{residuals =}\NormalTok{ gam_model}\OperatorTok{$}\NormalTok{residuals)}
\NormalTok{p10 <-}\StringTok{ }\KeywordTok{ggplot}\NormalTok{(}\DataTypeTok{data =}\NormalTok{ temp, }\KeywordTok{aes}\NormalTok{(}\DataTypeTok{x =}\NormalTok{ Time_fixed)) }\OperatorTok{+}
\StringTok{  }\KeywordTok{geom_line}\NormalTok{(}\KeywordTok{aes}\NormalTok{( }\DataTypeTok{y =}\NormalTok{ Influenza, }\DataTypeTok{color =} \StringTok{"Influenza"}\NormalTok{)) }\OperatorTok{+}
\StringTok{  }\KeywordTok{geom_line}\NormalTok{(}\KeywordTok{aes}\NormalTok{(}\DataTypeTok{y =}\NormalTok{ residuals, }\DataTypeTok{color =} \StringTok{"residuals"}\NormalTok{)) }\OperatorTok{+}
\StringTok{      }\KeywordTok{theme_light}\NormalTok{() }\OperatorTok{+}
\StringTok{  }\KeywordTok{scale_color_manual}\NormalTok{(}\DataTypeTok{values=}\KeywordTok{c}\NormalTok{(}\DataTypeTok{Influenza =} \StringTok{"#009E73"}\NormalTok{, }\DataTypeTok{residuals =} \StringTok{"#E69F00"}\NormalTok{)) }\OperatorTok{+}
\StringTok{  }\KeywordTok{labs}\NormalTok{(}\DataTypeTok{y =} \StringTok{"Influenza / Residual"}\NormalTok{) }\OperatorTok{+}
\StringTok{  }\KeywordTok{ggtitle}\NormalTok{(}\StringTok{"Plot of Influenza Residual vs. Time"}\NormalTok{)}
\NormalTok{p10}
\end{Highlighting}
\end{Shaded}

Analysis:With increase in Penalty factor the degrees of freedom should
decrease. Our results also confirm this relationship. Initially the
decline is steep. Some of the peaks in Influenza outbreaks correspond to
peaks in the residuals of the fitted model. Still,however, a lot of
variance in the residuals is not correlated to Influenza outbreaks.
Therefore, I would say that the Influenza outbreaks are not correlated
to the residuals.

\subsection{Question1.6:-6. Fit a GAM model in R in which mortality is
be modelled as an additive function of the spline functions of year,
week, and the number of confirmed cases of
influenza.}\label{question1.6-6.-fit-a-gam-model-in-r-in-which-mortality-is-be-modelled-as-an-additive-function-of-the-spline-functions-of-year-week-and-the-number-of-confirmed-cases-of-influenza.}

\begin{Shaded}
\begin{Highlighting}[]
\NormalTok{new_res=}\KeywordTok{gam}\NormalTok{(Mortality}\OperatorTok{~}\KeywordTok{s}\NormalTok{(Year,}\DataTypeTok{k=}\KeywordTok{length}\NormalTok{(}\KeywordTok{unique}\NormalTok{(influenza}\OperatorTok{$}\NormalTok{Year)))}
\OperatorTok{+}\KeywordTok{s}\NormalTok{(Week, }\DataTypeTok{k=}\KeywordTok{length}\NormalTok{(}\KeywordTok{unique}\NormalTok{(influenza}\OperatorTok{$}\NormalTok{Year))))this is continued up}
\KeywordTok{summary}\NormalTok{(new_res)}
\KeywordTok{plot}\NormalTok{(new_res, }\DataTypeTok{residuals=}\OtherTok{TRUE}\NormalTok{, }\DataTypeTok{page=}\DecValTok{1}\NormalTok{)}
\NormalTok{new_predicted <-}\StringTok{ }\KeywordTok{predict}\NormalTok{(new_res, }\DataTypeTok{newdata =}\NormalTok{ influenza, }\DataTypeTok{type=}\StringTok{'response'}\NormalTok{)}
\KeywordTok{ggplot}\NormalTok{(}\DataTypeTok{data =}\NormalTok{ influenza, }\KeywordTok{aes}\NormalTok{(}\DataTypeTok{x =}\NormalTok{ Time)) }\OperatorTok{+}
\KeywordTok{geom_point}\NormalTok{(}\KeywordTok{aes}\NormalTok{(}\DataTypeTok{y =}\NormalTok{ Mortality, }\DataTypeTok{colour =} \StringTok{"Mortality"}\NormalTok{)) }\OperatorTok{+}
\KeywordTok{geom_line}\NormalTok{(}\KeywordTok{aes}\NormalTok{(}\DataTypeTok{y =}\NormalTok{ new_predicted,}\DataTypeTok{colour =} \StringTok{"New Model"}\NormalTok{)) }\OperatorTok{+}
\KeywordTok{geom_line}\NormalTok{(}\KeywordTok{aes}\NormalTok{(}\DataTypeTok{y =}\NormalTok{ predicted,}\DataTypeTok{colour =} \StringTok{"Previous Model"}\NormalTok{)) }\OperatorTok{+}
\KeywordTok{scale_colour_manual}\NormalTok{(}\StringTok{""}\NormalTok{, }\DataTypeTok{breaks =} \KeywordTok{c}\NormalTok{(}\StringTok{"Mortality"}\NormalTok{,}\StringTok{"New Model"}\NormalTok{, }\StringTok{"Previous Model"}\NormalTok{),}
\DataTypeTok{values =} \KeywordTok{c}\NormalTok{(}\StringTok{"Blue"}\NormalTok{, }\StringTok{"Red"}\NormalTok{, }\StringTok{"Green"}\NormalTok{))}
\CommentTok{#(or)}
\NormalTok{gam_model_additive <-}\StringTok{ }\NormalTok{mgcv}\OperatorTok{::}\KeywordTok{gam}\NormalTok{(}\DataTypeTok{data =}\NormalTok{ flu_data, Mortality}\OperatorTok{~}\KeywordTok{s}\NormalTok{(Year)}\OperatorTok{+}\KeywordTok{s}\NormalTok{(Week), }\DataTypeTok{method =} \StringTok{"GCV.Cp"}\NormalTok{)}
\NormalTok{k1 =}\StringTok{ }\KeywordTok{length}\NormalTok{(}\KeywordTok{unique}\NormalTok{(flu_data}\OperatorTok{$}\NormalTok{Year))}
\NormalTok{k2 =}\StringTok{ }\KeywordTok{length}\NormalTok{(}\KeywordTok{unique}\NormalTok{(flu_data}\OperatorTok{$}\NormalTok{Week))}
\NormalTok{k3 =}\StringTok{ }\KeywordTok{length}\NormalTok{(}\KeywordTok{unique}\NormalTok{(flu_data}\OperatorTok{$}\NormalTok{Influenza))}
\NormalTok{gam_model_additive <-}\StringTok{ }\KeywordTok{gam}\NormalTok{(Mortality }\OperatorTok{~}\StringTok{ }\KeywordTok{s}\NormalTok{(Year, }\DataTypeTok{k=}\NormalTok{k1) }\OperatorTok{+}
\StringTok{                                     }\KeywordTok{s}\NormalTok{(Week, }\DataTypeTok{k=}\NormalTok{k2) }\OperatorTok{+}
\StringTok{                                    }\KeywordTok{s}\NormalTok{(Influenza, }\DataTypeTok{k=}\NormalTok{k3),}
                          \DataTypeTok{data =}\NormalTok{ flu_data)}
\KeywordTok{summary}\NormalTok{(gam_model_additive)}
\NormalTok{flu_data}\OperatorTok{$}\NormalTok{fitted.values =}\StringTok{ }\NormalTok{gam_model_additive}\OperatorTok{$}\NormalTok{fitted.values}

\NormalTok{p11 <-}\StringTok{ }\KeywordTok{ggplot}\NormalTok{(}\DataTypeTok{data =}\NormalTok{ flu_data, }\KeywordTok{aes}\NormalTok{(}\DataTypeTok{x =}\NormalTok{ Time_fixed)) }\OperatorTok{+}
\StringTok{  }\KeywordTok{geom_line}\NormalTok{(}\KeywordTok{aes}\NormalTok{( }\DataTypeTok{y =}\NormalTok{ Mortality, }\DataTypeTok{color =} \StringTok{"Mortality"}\NormalTok{)) }\OperatorTok{+}
\StringTok{  }\KeywordTok{geom_line}\NormalTok{(}\KeywordTok{aes}\NormalTok{(}\DataTypeTok{y =}\NormalTok{ fitted.values, }\DataTypeTok{color =} \StringTok{"fitted.values"}\NormalTok{)) }\OperatorTok{+}
\StringTok{      }\KeywordTok{theme_light}\NormalTok{() }\OperatorTok{+}
\StringTok{  }\KeywordTok{scale_color_manual}\NormalTok{(}\DataTypeTok{values=}\KeywordTok{c}\NormalTok{(}\DataTypeTok{Mortality =} \StringTok{"#009E73"}\NormalTok{, }\DataTypeTok{fitted.values =} \StringTok{"#E69F00"}\NormalTok{)) }\OperatorTok{+}
\StringTok{  }\KeywordTok{labs}\NormalTok{(}\DataTypeTok{y =} \StringTok{"Mortality / fitted.values"}\NormalTok{) }\OperatorTok{+}
\StringTok{  }\KeywordTok{ggtitle}\NormalTok{(}\StringTok{"Plot of Mortality and Fitted vs. Time"}\NormalTok{)}
\NormalTok{p11}
\end{Highlighting}
\end{Shaded}

Analysis:-From the p values, we can conclude that Influenza cases have
high impact on mortality The additive GAM model clearly has the best
fit. Much of the variance of the data (81.9\% is the adjusted R\^{}2) is
captured by the model. Given that the GAM models in step 2 and step 4 do
not include the influenza variable from the dataset, and the the model
above does, one can say that most likely mortality is influenced by the
outbreaks of influenza.

\section{Assignment 2. High-dimensional
methods}\label{assignment-2.-high-dimensional-methods}

The data file data.csv contains information about 64 e-mails which were
manually collected from DBWorld mailing list. They were classified as:
`announces of conferences' (1) and `everything else' (0) (variable
Conference) 1. Divide data into training and test sets (70/30) without
scaling. Perform nearest shrunken centroid classification of training
data in which the threshold is chosen by cross-validation. Provide a
centroid plot and interpret it. How many features were selected by the
method? List the names of the 10 most contributing features and comment
whether it is reasonable that they have strong effect on the
discrimination between the conference mails and other mails? Report the
test error. 2. Compute the test error and the number of the contributing
features for the following methods fitted to the training data: a.
Elastic net with the binomial response and \(\alpha\) ????????=0.5 in
which penalty is selected by the cross-validation b. Support vector
machine with ``vanilladot'' kernel. Compare the results of these models
with the results of the nearest shrunken centroids (make a comparative
table). Which model would you prefer and why? 3. Implement
Benjamini-Hochberg method for the original data, and use t.test() for
computing p-values. Which features correspond to the rejected
hypotheses? Interpret the result

\subsection{Question2.1:- Perform nearest shrunken centroid
classification}\label{question2.1--perform-nearest-shrunken-centroid-classification}

\begin{Shaded}
\begin{Highlighting}[]
\KeywordTok{library}\NormalTok{(pamr)}
\KeywordTok{library}\NormalTok{(glmnet)}
\KeywordTok{library}\NormalTok{(dplyr)}
\KeywordTok{library}\NormalTok{(kernlab)}
\CommentTok{# dividing data into train and test set}
\NormalTok{data <-}\StringTok{ }\KeywordTok{read.csv}\NormalTok{(}\DataTypeTok{file =} \StringTok{"data.csv"}\NormalTok{, }\DataTypeTok{sep =} \StringTok{";"}\NormalTok{, }\DataTypeTok{header =} \OtherTok{TRUE}\NormalTok{, }\DataTypeTok{fileEncoding =} \StringTok{"Latin1"}\NormalTok{)}
\NormalTok{data}\OperatorTok{$}\NormalTok{Conference=}\KeywordTok{as.factor}\NormalTok{(data}\OperatorTok{$}\NormalTok{Conference)}
\KeywordTok{rownames}\NormalTok{(data)=}\DecValTok{1}\OperatorTok{:}\KeywordTok{nrow}\NormalTok{(data)}
\NormalTok{n=}\KeywordTok{dim}\NormalTok{(data)[}\DecValTok{1}\NormalTok{]}
\KeywordTok{set.seed}\NormalTok{(}\DecValTok{12345}\NormalTok{)}
\NormalTok{id=}\KeywordTok{sample}\NormalTok{(}\DecValTok{1}\OperatorTok{:}\NormalTok{n, }\KeywordTok{floor}\NormalTok{(n}\OperatorTok{*}\FloatTok{0.7}\NormalTok{))}\CommentTok{#0.7 is 70%}
\NormalTok{train=data[id,]}
\NormalTok{test=data[}\OperatorTok{-}\NormalTok{id,]}
\CommentTok{#Perform of training data in which the threshold is chosen}
\CommentTok{# by cross-validation}
\NormalTok{x =}\StringTok{ }\KeywordTok{t}\NormalTok{(train[,}\OperatorTok{-}\DecValTok{4703}\NormalTok{])}\CommentTok{#4703 are the variables in the data}
\NormalTok{y =}\StringTok{ }\NormalTok{train[[}\DecValTok{4703}\NormalTok{]]}
\NormalTok{x_test =}\StringTok{ }\KeywordTok{t}\NormalTok{(test[,}\OperatorTok{-}\DecValTok{4703}\NormalTok{])}
\NormalTok{y_test =}\StringTok{ }\NormalTok{test[[}\DecValTok{4703}\NormalTok{]]}
\NormalTok{mydata=}\KeywordTok{list}\NormalTok{(}\DataTypeTok{x=}\NormalTok{x,}\DataTypeTok{y=}\KeywordTok{as.factor}\NormalTok{(y),}\DataTypeTok{geneid=}\KeywordTok{as.character}\NormalTok{(}\DecValTok{1}\OperatorTok{:}\KeywordTok{nrow}\NormalTok{(x)), }\DataTypeTok{genenames=}\KeywordTok{rownames}\NormalTok{(x))}
\NormalTok{model =}\StringTok{ }\KeywordTok{pamr.train}\NormalTok{(mydata,}\DataTypeTok{threshold=}\KeywordTok{seq}\NormalTok{(}\DecValTok{0}\NormalTok{,}\DecValTok{4}\NormalTok{, }\FloatTok{0.1}\NormalTok{))}
\NormalTok{cvmodel=}\KeywordTok{pamr.cv}\NormalTok{(model,mydata)}
\KeywordTok{pamr.plotcv}\NormalTok{(cvmodel)}
\CommentTok{# from cv model,when the threshold is 1.3 and 1.4, the error is least. Hence selecting the threshold as #centroid plot}
\KeywordTok{pamr.plotcen}\NormalTok{(model, mydata, }\DataTypeTok{threshold=}\FloatTok{1.4}\NormalTok{)}
\CommentTok{# The method selected 231 features}
\NormalTok{contri_genes=}\KeywordTok{pamr.listgenes}\NormalTok{(model,mydata,}\DataTypeTok{threshold=}\FloatTok{1.4}\NormalTok{)}
\CommentTok{# 10 most contributing features}
\NormalTok{imp =}\StringTok{ }\KeywordTok{as.numeric}\NormalTok{(contri_genes[}\DecValTok{1}\OperatorTok{:}\DecValTok{10}\NormalTok{,}\DecValTok{1}\NormalTok{])}
\KeywordTok{cat}\NormalTok{( }\KeywordTok{paste}\NormalTok{( }\KeywordTok{colnames}\NormalTok{(data[imp]), }\DataTypeTok{collapse=}\StringTok{'}\CharTok{\textbackslash{}n}\StringTok{'}\NormalTok{ ) )}
\CommentTok{#test error}
\NormalTok{pred_test <-}\StringTok{ }\KeywordTok{pamr.predict}\NormalTok{(model, }\DataTypeTok{newx =}\NormalTok{ x_test, }\DataTypeTok{threshold=}\FloatTok{1.4}\NormalTok{)}
\NormalTok{misclass_table <-}\StringTok{ }\KeywordTok{table}\NormalTok{(y_test,pred_test)}
\NormalTok{test_error <-}\StringTok{ }\DecValTok{1} \OperatorTok{-}\StringTok{ }\KeywordTok{sum}\NormalTok{(}\KeywordTok{diag}\NormalTok{(misclass_table))}\OperatorTok{/}\KeywordTok{sum}\NormalTok{(misclass_table)}
\CommentTok{#Elastic net}
\NormalTok{x =}\StringTok{ }\NormalTok{train[,}\OperatorTok{-}\DecValTok{4703}\NormalTok{]}\OperatorTok\StringTok{ }\KeywordTok{as.matrix}\NormalTok{()}
\NormalTok{y =}\StringTok{ }\NormalTok{train[[}\DecValTok{4703}\NormalTok{]]}
\NormalTok{x_test =}\StringTok{ }\NormalTok{test[,}\OperatorTok{-}\DecValTok{4703}\NormalTok{]}\OperatorTok\StringTok{ }\KeywordTok{as.matrix}\NormalTok{()}
\NormalTok{y_test =}\StringTok{ }\NormalTok{test[[}\DecValTok{4703}\NormalTok{]]}
\NormalTok{cvfit <-}\StringTok{ }\KeywordTok{cv.glmnet}\NormalTok{(x, y, }\DataTypeTok{family =} \StringTok{"binomial"}\NormalTok{, }\DataTypeTok{alpha =} \FloatTok{0.5}\NormalTok{)}
\KeywordTok{plot}\NormalTok{(cvfit)}
\NormalTok{predict_elastic <-}\StringTok{ }\KeywordTok{predict.cv.glmnet}\NormalTok{(cvfit, }\DataTypeTok{newx =}\NormalTok{ x_test, }\DataTypeTok{s=}\StringTok{"lambda.min"}\NormalTok{,}\DataTypeTok{type =} \StringTok{"class"}\NormalTok{)}
\NormalTok{coeffs <-}\StringTok{ }\KeywordTok{coef}\NormalTok{(cvfit,}\DataTypeTok{s=}\StringTok{"lambda.min"}\NormalTok{)}
\NormalTok{e_variables =}\StringTok{ }\KeywordTok{as.data.frame}\NormalTok{(coeffs[}\KeywordTok{which}\NormalTok{(coeffs[,}\DecValTok{1}\NormalTok{]}\OperatorTok{!=}\DecValTok{0}\NormalTok{),])}
\NormalTok{cmatrix_elastic <-}\StringTok{ }\KeywordTok{table}\NormalTok{(y_test, predict_elastic)}
\NormalTok{testerror_elastic <-}\StringTok{ }\DecValTok{1} \OperatorTok{-}\KeywordTok{sum}\NormalTok{(}\KeywordTok{diag}\NormalTok{(cmatrix_elastic))}\OperatorTok{/}\KeywordTok{sum}\NormalTok{(cmatrix_elastic)}
\end{Highlighting}
\end{Shaded}

From plot of cross validation model, which shows the depandance of
misclassification error vs the threshold value, we can see that the
misclassification error is least when the threshold is 1.3 and 1.4. So
for this analysis, threshold value of 1.4 is chosen. Nearest centroid
method computes a standardized centroid for each class. The nearest
shrunken centroid classification ``shrinks'' the class centroids towards
zero by threshold, setting it equal to zero if it hits zero. After the
shrinkage, the new sample is classifies using the centroid rule. So this
will make the classifier more accurate by reducing the effecte of
unimportant features so feature selection happens here. Here the
centroid plot shows the shrunken centroids for each of the variables
which has non zero differance to the shrunken centroids of two classes.
The method selected 231 features. The 10 most contributing features are:
paper, important, submissions, due, published, position, call,
conferance, dates, candidates All the names and features mentioned above
are significant in classification of emails as conferance emails or not.
Test Error : 0.1 - 10\% test error and 90\% accuracy.

\subsection{Question2.2:-SVM}\label{question2.2-svm}

\begin{Shaded}
\begin{Highlighting}[]
\CommentTok{#SVM}
\NormalTok{svm_model <-}\StringTok{ }\KeywordTok{ksvm}\NormalTok{(x, y, }\DataTypeTok{type =} \StringTok{'C-svc'}\NormalTok{, }\DataTypeTok{kernel =} \StringTok{"vanilladot"}\NormalTok{, }\DataTypeTok{scale =}\NormalTok{ F)}
\NormalTok{predict_svm <-}\StringTok{ }\KeywordTok{predict}\NormalTok{(svm_model,}\DataTypeTok{newdata =}\NormalTok{ x_test, }\DataTypeTok{type =} \StringTok{"response"}\NormalTok{)}
\NormalTok{mtable_svm <-}\StringTok{ }\KeywordTok{table}\NormalTok{(y_test,predict_svm)}
\NormalTok{testerror_svm <-}\StringTok{ }\DecValTok{1} \OperatorTok{-}\KeywordTok{sum}\NormalTok{(}\KeywordTok{diag}\NormalTok{(mtable_svm))}\OperatorTok{/}\KeywordTok{sum}\NormalTok{(mtable_svm)}
\CommentTok{# table}
\NormalTok{table_result <-}\StringTok{ }\KeywordTok{data.frame}\NormalTok{(}\StringTok{"Model"}\NormalTok{ =}\StringTok{ }\KeywordTok{c}\NormalTok{(}\StringTok{"Nearest Shrunken Centroid Model"}\NormalTok{,}
\StringTok{"ElasticNet Model"}\NormalTok{, }\StringTok{"SVM Model"}\NormalTok{), }\StringTok{"Features"}\NormalTok{ =}\StringTok{ }\KeywordTok{c}\NormalTok{(}\DecValTok{231}\NormalTok{,}\DecValTok{40}\NormalTok{,}\DecValTok{41}\NormalTok{),}
\StringTok{"Error"}\NormalTok{ =}\StringTok{ }\KeywordTok{c}\NormalTok{(test_error, testerror_elastic,testerror_svm ))}
\NormalTok{knitr}\OperatorTok{::}\KeywordTok{kable}\NormalTok{(table_result, }\DataTypeTok{caption =} \StringTok{"Model Comparison"}\NormalTok{)}
\end{Highlighting}
\end{Shaded}

On comparing all three models, Nearest Shrunken Centeroid model and
Elastic model has same rate of test error which is 0.1 (10\%) where
centeroid model takes 231 parameters as contributing parameters whereas
Elastic net model makes use of less number of features to do the
prediction as the contributing features. In case of SVM model, it uses
41 features as contributing features and the error rate is 0.05 or 5\%
which is less than the above two models and it uses 41 features as
contributing features which is not significantly differant compared to
the elastic model. So based on the above two factors, SVM model is
better for classification of the data because of the accuracy that the
model provides with less number of features

\section{Question:-2.3-Benjamini-Hochberg
method}\label{question-2.3-benjamini-hochberg-method}

\begin{Shaded}
\begin{Highlighting}[]
\NormalTok{pvalues <-}\StringTok{ }\KeywordTok{c}\NormalTok{()}
\NormalTok{x <-}\StringTok{ }\NormalTok{data[,}\OperatorTok{-}\DecValTok{4703}\NormalTok{]}
\CommentTok{#y <- as.vector(data[,4703])}
\ControlFlowTok{for}\NormalTok{(i }\ControlFlowTok{in} \DecValTok{1}\OperatorTok{:}\NormalTok{(}\KeywordTok{ncol}\NormalTok{(data)}\OperatorTok{-}\DecValTok{1}\NormalTok{))\{}
\NormalTok{t_res <-}\StringTok{ }\KeywordTok{t.test}\NormalTok{(x[,i]}\OperatorTok{~}\NormalTok{Conference, }\DataTypeTok{data =}\NormalTok{ data)}
\DecValTok{14}
\NormalTok{pvalues[i] <-}\StringTok{ }\NormalTok{t_res}\OperatorTok{$}\NormalTok{p.value}
\NormalTok{\}}
\NormalTok{p_adj <-}\StringTok{ }\KeywordTok{p.adjust}\NormalTok{(pvalues, }\DataTypeTok{method =} \StringTok{"BH"}\NormalTok{, }\DataTypeTok{n =} \KeywordTok{length}\NormalTok{(pvalues))}
\NormalTok{p_adj_df <-}\StringTok{ }\KeywordTok{data.frame}\NormalTok{(}\StringTok{"feature"}\NormalTok{ =}\StringTok{ }\KeywordTok{colnames}\NormalTok{(data[,}\OperatorTok{-}\DecValTok{4703}\NormalTok{]), }\StringTok{"pvals"}\NormalTok{ =}\StringTok{ }\NormalTok{p_adj)}
\NormalTok{p_adj_df <-}\StringTok{ }\NormalTok{p_adj_df[}\KeywordTok{which}\NormalTok{(p_adj_df[,}\DecValTok{2}\NormalTok{] }\OperatorTok{<=}\StringTok{ }\FloatTok{0.05}\NormalTok{), ]}
\NormalTok{num_features <-}\StringTok{ }\KeywordTok{nrow}\NormalTok{(p_adj_df)}
\NormalTok{p_adj_df[}\KeywordTok{c}\NormalTok{(}\DecValTok{1}\OperatorTok{:}\DecValTok{39}\NormalTok{),}\DecValTok{1}\NormalTok{]}
\end{Highlighting}
\end{Shaded}

We consideres null hypothesis as the feature is not significant for
analysing if the email is a conferance email and alternate hopothesis:
The feature is significant in classifying coferance email. From the
above analysis, we can see that the model selected 39 features as
significant features by rejecting the null hypothesis(the features with
p values less than 0.05). The features corresponding to rejected
hypothesis are: {[}1{]} apply authors call camera candidate candidates
chairs \#\# {[}36{]} submission team topics workshop 4702 Levels: a4soa
aaai aachen aalborg aamas aarhus aaron aau abbadi abdalla abdallah
abductive abilities The above features are the significant features as
selected by the model. From the above feature list, the words team,
important, topics, presented, proceedings, helg, org, international,
call, papers, phd, published etc seems very relevant in classification
of a conferance email

\section{Assignment 2-2method}\label{assignment-2-2method}

\subsection{2.1.}\label{section}

\begin{Shaded}
\begin{Highlighting}[]
\KeywordTok{rm}\NormalTok{(}\DataTypeTok{list=}\KeywordTok{ls}\NormalTok{())}
\KeywordTok{gc}\NormalTok{()}
\NormalTok{data <-}\StringTok{ }\KeywordTok{read.csv}\NormalTok{(}\DataTypeTok{file =} \StringTok{"data.csv"}\NormalTok{, }\DataTypeTok{sep =} \StringTok{";"}\NormalTok{, }\DataTypeTok{header =} \OtherTok{TRUE}\NormalTok{)}
\end{Highlighting}
\end{Shaded}

\begin{Shaded}
\begin{Highlighting}[]
\NormalTok{n=}\KeywordTok{NROW}\NormalTok{(data)}
\NormalTok{data}\OperatorTok{$}\NormalTok{Conference <-}\StringTok{ }\KeywordTok{as.factor}\NormalTok{(data}\OperatorTok{$}\NormalTok{Conference)}
\KeywordTok{set.seed}\NormalTok{(}\DecValTok{12345}\NormalTok{)}
\NormalTok{id=}\KeywordTok{sample}\NormalTok{(}\DecValTok{1}\OperatorTok{:}\NormalTok{n, }\KeywordTok{floor}\NormalTok{(n}\OperatorTok{*}\FloatTok{0.7}\NormalTok{))}
\NormalTok{train=data[id,]}
\NormalTok{test =}\StringTok{ }\NormalTok{data[}\OperatorTok{-}\NormalTok{id,]}
\KeywordTok{rownames}\NormalTok{(train)=}\DecValTok{1}\OperatorTok{:}\KeywordTok{nrow}\NormalTok{(train)}
\NormalTok{x=}\KeywordTok{t}\NormalTok{(train[,}\OperatorTok{-}\DecValTok{4703}\NormalTok{])}
\NormalTok{y=train[[}\DecValTok{4703}\NormalTok{]]}
\KeywordTok{rownames}\NormalTok{(test)=}\DecValTok{1}\OperatorTok{:}\KeywordTok{nrow}\NormalTok{(test)}
\NormalTok{x_test=}\KeywordTok{t}\NormalTok{(test[,}\OperatorTok{-}\DecValTok{4703}\NormalTok{])}
\NormalTok{y_test=test[[}\DecValTok{4703}\NormalTok{]]}
\NormalTok{mydata =}\StringTok{ }\KeywordTok{list}\NormalTok{(}\DataTypeTok{x=}\NormalTok{x,}\DataTypeTok{y=}\KeywordTok{as.factor}\NormalTok{(y),}\DataTypeTok{geneid=}\KeywordTok{as.character}\NormalTok{(}\DecValTok{1}\OperatorTok{:}\KeywordTok{nrow}\NormalTok{(x)), }\DataTypeTok{genenames=}\KeywordTok{rownames}\NormalTok{(x))}
\NormalTok{mydata_test =}\StringTok{ }\KeywordTok{list}\NormalTok{(}\DataTypeTok{x=}\NormalTok{x_test,}\DataTypeTok{y=}\KeywordTok{as.factor}\NormalTok{(y_test),}\DataTypeTok{geneid=}\KeywordTok{as.character}\NormalTok{(}\DecValTok{1}\OperatorTok{:}\KeywordTok{nrow}\NormalTok{(x)), }\DataTypeTok{genenames=}\KeywordTok{rownames}\NormalTok{(x))}
\NormalTok{model=}\KeywordTok{pamr.train}\NormalTok{(mydata,}\DataTypeTok{threshold=}\KeywordTok{seq}\NormalTok{(}\DecValTok{0}\NormalTok{, }\DecValTok{4}\NormalTok{, }\FloatTok{0.1}\NormalTok{))}
\NormalTok{cvmodel=}\KeywordTok{pamr.cv}\NormalTok{(model, mydata)}
\NormalTok{important_gen <-}\StringTok{ }\KeywordTok{as.data.frame}\NormalTok{(}\KeywordTok{pamr.listgenes}\NormalTok{(model, mydata, }\DataTypeTok{threshold =} \FloatTok{1.3}\NormalTok{))}
\NormalTok{predicted_scc_test <-}\StringTok{ }\KeywordTok{pamr.predict}\NormalTok{(model, }\DataTypeTok{newx =}\NormalTok{ x_test, }\DataTypeTok{threshold =} \FloatTok{1.3}\NormalTok{)}
\end{Highlighting}
\end{Shaded}

\begin{Shaded}
\begin{Highlighting}[]
\NormalTok{### plots}
\KeywordTok{pamr.plotcv}\NormalTok{(cvmodel)}
\KeywordTok{pamr.plotcen}\NormalTok{(model, mydata, }\DataTypeTok{threshold =} \FloatTok{1.3}\NormalTok{)}
\end{Highlighting}
\end{Shaded}

\begin{Shaded}
\begin{Highlighting}[]
\NormalTok{### important features}
\NormalTok{## List the significant genes}
\KeywordTok{NROW}\NormalTok{(important_gen)}
\NormalTok{temp <-}\StringTok{ }\KeywordTok{colnames}\NormalTok{(data) }\OperatorTok\StringTok{ }\KeywordTok{as.data.frame}\NormalTok{()}
\KeywordTok{colnames}\NormalTok{(temp) <-}\StringTok{ "col_name"}
\NormalTok{temp}\OperatorTok{$}\NormalTok{index <-}\StringTok{ }\KeywordTok{row.names}\NormalTok{(temp)}
\NormalTok{df <-}\StringTok{ }\KeywordTok{merge}\NormalTok{(}\DataTypeTok{x =}\NormalTok{ important_gen, }\DataTypeTok{y =}\NormalTok{ temp, }\DataTypeTok{by.x =} \StringTok{"id"}\NormalTok{, }\DataTypeTok{by.y =} \StringTok{"index"}\NormalTok{, }\DataTypeTok{all.x =} \OtherTok{TRUE}\NormalTok{)}
\NormalTok{df <-}\StringTok{ }\NormalTok{df[}\KeywordTok{order}\NormalTok{(df[,}\DecValTok{3}\NormalTok{], }\DataTypeTok{decreasing =} \OtherTok{TRUE}\NormalTok{ ),]}
\CommentTok{#knitr::kable(head(df[,4],10), caption = "Important feaures selected by Nearest Shrunken Centroids ")}
\end{Highlighting}
\end{Shaded}

\begin{Shaded}
\begin{Highlighting}[]
\NormalTok{### confusion table}
\KeywordTok{library}\NormalTok{(caret)}
\NormalTok{conf_scc <-}\StringTok{ }\KeywordTok{table}\NormalTok{(y_test, predicted_scc_test)}
\KeywordTok{names}\NormalTok{(}\KeywordTok{dimnames}\NormalTok{(conf_scc)) <-}\StringTok{ }\KeywordTok{c}\NormalTok{(}\StringTok{"Actual Test"}\NormalTok{, }\StringTok{"Predicted Srunken Centroid Test"}\NormalTok{)}
\NormalTok{result_scc <-}\StringTok{ }\NormalTok{caret}\OperatorTok{::}\KeywordTok{confusionMatrix}\NormalTok{(conf_scc)}
\CommentTok{#caret::confusionMatrix(conf_scc)}
\end{Highlighting}
\end{Shaded}

\subsection{2.2}\label{section-1}

\begin{Shaded}
\begin{Highlighting}[]
\NormalTok{x =}\StringTok{ }\NormalTok{train[,}\OperatorTok{-}\DecValTok{4703}\NormalTok{] }\OperatorTok\StringTok{ }\KeywordTok{as.matrix}\NormalTok{()}
\NormalTok{y =}\StringTok{ }\NormalTok{train[,}\DecValTok{4703}\NormalTok{]}
\NormalTok{x_test =}\StringTok{ }\NormalTok{test[,}\OperatorTok{-}\DecValTok{4703}\NormalTok{] }\OperatorTok\StringTok{ }\KeywordTok{as.matrix}\NormalTok{()}
\NormalTok{y_test =}\StringTok{ }\NormalTok{test[,}\DecValTok{4703}\NormalTok{]}
\NormalTok{cvfit =}\StringTok{ }\KeywordTok{cv.glmnet}\NormalTok{(}\DataTypeTok{x=}\NormalTok{x, }\DataTypeTok{y=}\NormalTok{y, }\DataTypeTok{alpha =} \FloatTok{0.5}\NormalTok{, }\DataTypeTok{family =}   \StringTok{"binomial"}\NormalTok{)}
\NormalTok{predicted_elastic_test <-}\StringTok{ }\KeywordTok{predict.cv.glmnet}\NormalTok{(cvfit, }\DataTypeTok{newx =}\NormalTok{ x_test, }\DataTypeTok{s =} \StringTok{"lambda.min"}\NormalTok{, }\DataTypeTok{type =} \StringTok{"class"}\NormalTok{)}
\NormalTok{tmp_coeffs <-}\StringTok{ }\KeywordTok{coef}\NormalTok{(cvfit, }\DataTypeTok{s =} \StringTok{"lambda.min"}\NormalTok{)}
\NormalTok{elastic_variable <-}\StringTok{ }\KeywordTok{data.frame}\NormalTok{(}\DataTypeTok{name =}\NormalTok{ tmp_coeffs}\OperatorTok{@}\NormalTok{Dimnames[[}\DecValTok{1}\NormalTok{]][tmp_coeffs}\OperatorTok{@}\NormalTok{i }\OperatorTok{+}\StringTok{ }\DecValTok{1}\NormalTok{], }\DataTypeTok{coefficient =}\NormalTok{ tmp_coeffs}\OperatorTok{@}\NormalTok{x)}
\CommentTok{#knitr::kable(elastic_variable, caption = "Contributing features in the elastic model")}
\NormalTok{conf_elastic_net <-}\StringTok{ }\KeywordTok{table}\NormalTok{(y_test, predicted_elastic_test)}
\KeywordTok{names}\NormalTok{(}\KeywordTok{dimnames}\NormalTok{(conf_elastic_net)) <-}\StringTok{ }\KeywordTok{c}\NormalTok{(}\StringTok{"Actual Test"}\NormalTok{, }\StringTok{"Predicted ElasticNet Test"}\NormalTok{)}
\NormalTok{result_elastic_net <-}\StringTok{ }\NormalTok{caret}\OperatorTok{::}\KeywordTok{confusionMatrix}\NormalTok{(conf_elastic_net)}
\CommentTok{#caret::confusionMatrix(conf_elastic_net)}
\CommentTok{# svm}
\NormalTok{svm_fit <-}\StringTok{ }\NormalTok{kernlab}\OperatorTok{::}\KeywordTok{ksvm}\NormalTok{(x, y, }\DataTypeTok{kernel=}\StringTok{"vanilladot"}\NormalTok{, }\DataTypeTok{scale =} \OtherTok{FALSE}\NormalTok{, }\DataTypeTok{type =} \StringTok{"C-svc"}\NormalTok{)}
\NormalTok{predicted_svm_test <-}\StringTok{ }\KeywordTok{predict}\NormalTok{(svm_fit, x_test, }\DataTypeTok{type=}\StringTok{"response"}\NormalTok{)}
\NormalTok{conf_svm_tree <-}\StringTok{ }\KeywordTok{table}\NormalTok{(y_test, predicted_svm_test)}
\KeywordTok{names}\NormalTok{(}\KeywordTok{dimnames}\NormalTok{(conf_svm_tree)) <-}\StringTok{ }\KeywordTok{c}\NormalTok{(}\StringTok{"Actual Test"}\NormalTok{, }\StringTok{"Predicted SVM Test"}\NormalTok{)}
\NormalTok{result_svm <-}\StringTok{ }\NormalTok{caret}\OperatorTok{::}\KeywordTok{confusionMatrix}\NormalTok{(conf_svm_tree)}
\CommentTok{#caret::confusionMatrix(conf_svm_tree)}
\CommentTok{# creating table}
\NormalTok{final_result <-}\StringTok{ }\KeywordTok{cbind}\NormalTok{(result_scc}\OperatorTok{$}\NormalTok{overall[[}\DecValTok{1}\NormalTok{]]}\OperatorTok{*}\DecValTok{100}\NormalTok{,}
\NormalTok{                      result_elastic_net}\OperatorTok{$}\NormalTok{overall[[}\DecValTok{1}\NormalTok{]]}\OperatorTok{*}\DecValTok{100}\NormalTok{,}
\NormalTok{                      result_svm}\OperatorTok{$}\NormalTok{overall[[}\DecValTok{1}\NormalTok{]] }\OperatorTok{*}\DecValTok{100}\NormalTok{) }\OperatorTok\StringTok{ }\KeywordTok{as.data.frame}\NormalTok{()}
\NormalTok{features_count <-}\StringTok{ }\KeywordTok{cbind}\NormalTok{(}\KeywordTok{NROW}\NormalTok{(important_gen), }\KeywordTok{NROW}\NormalTok{(elastic_variable), }\KeywordTok{NCOL}\NormalTok{(data))}
\NormalTok{final_result <-}\StringTok{ }\KeywordTok{rbind}\NormalTok{(final_result, features_count)}
\KeywordTok{colnames}\NormalTok{(final_result) <-}\StringTok{ }\KeywordTok{c}\NormalTok{(}\StringTok{"Nearest Shrunken Centroid Model"}\NormalTok{,}
                            \StringTok{"ElasticNet Model"}\NormalTok{, }\StringTok{"SVM Model"}\NormalTok{)}
\KeywordTok{rownames}\NormalTok{(final_result) <-}\StringTok{ }\KeywordTok{c}\NormalTok{(}\StringTok{"Accuracy"}\NormalTok{, }\StringTok{"Number of Features"}\NormalTok{)}
\CommentTok{#knitr::kable(final_result, caption = "Comparsion of Models on Test dataset")}
\end{Highlighting}
\end{Shaded}

\subsection{2.3. Implement Benjamini-Hochberg method for the original
data, and use t.test() for computing p-values. Which features correspond
to the rejected hypotheses? Interpret the
result.}\label{implement-benjamini-hochberg-method-for-the-original-data-and-use-t.test-for-computing-p-values.-which-features-correspond-to-the-rejected-hypotheses-interpret-the-result.}

\begin{Shaded}
\begin{Highlighting}[]
\NormalTok{y <-}\StringTok{ }\KeywordTok{as.factor}\NormalTok{(data[,}\DecValTok{4703}\NormalTok{])}
\NormalTok{x <-}\StringTok{ }\KeywordTok{as.matrix}\NormalTok{(data[,}\OperatorTok{-}\DecValTok{4703}\NormalTok{])}
\NormalTok{p_values <-}\StringTok{ }\KeywordTok{data.frame}\NormalTok{(}\DataTypeTok{feature =} \StringTok{''}\NormalTok{,}\DataTypeTok{P_value =} \DecValTok{0}\NormalTok{,}\DataTypeTok{stringsAsFactors =} \OtherTok{FALSE}\NormalTok{)}
\ControlFlowTok{for}\NormalTok{(i }\ControlFlowTok{in} \DecValTok{1}\OperatorTok{:}\KeywordTok{ncol}\NormalTok{(x))\{}
\NormalTok{res =}\StringTok{ }\KeywordTok{t.test}\NormalTok{(x[,i]}\OperatorTok{~}\NormalTok{y, }\DataTypeTok{data =}\NormalTok{ data,}
\DataTypeTok{alternative=}\StringTok{"two.sided"}
\NormalTok{,}\DataTypeTok{conf.level =} \FloatTok{0.95}\NormalTok{)}
\NormalTok{p_values[i,] <-}\StringTok{ }\KeywordTok{c}\NormalTok{(}\KeywordTok{colnames}\NormalTok{(x)[i],res}\OperatorTok{$}\NormalTok{p.value)}
\NormalTok{\}}
\NormalTok{p_values}\OperatorTok{$}\NormalTok{P_value <-}\StringTok{ }\KeywordTok{as.numeric}\NormalTok{(p_values}\OperatorTok{$}\NormalTok{P_value)}
\NormalTok{p <-}\StringTok{ }\KeywordTok{p.adjust}\NormalTok{(p_values}\OperatorTok{$}\NormalTok{P_value, }\DataTypeTok{method =} \StringTok{'BH'}\NormalTok{)}
\KeywordTok{length}\NormalTok{(p[}\KeywordTok{which}\NormalTok{(p }\OperatorTok{>}\StringTok{ }\FloatTok{0.05}\NormalTok{)])}
\NormalTok{out <-}\StringTok{ }\NormalTok{p_values[}\KeywordTok{which}\NormalTok{(p }\OperatorTok{<=}\StringTok{ }\FloatTok{0.05}\NormalTok{),]}
\NormalTok{out <-}\StringTok{ }\NormalTok{out[}\KeywordTok{order}\NormalTok{(out}\OperatorTok{$}\NormalTok{P_value),]}
\KeywordTok{rownames}\NormalTok{(out) <-}\StringTok{ }\OtherTok{NULL}
\CommentTok{#out}
\end{Highlighting}
\end{Shaded}


\end{document}
